\documentclass[10pt]{article}
\usepackage[utf8]{inputenc}
\usepackage{amsfonts}% or latexsym, amssymb, mathabx, txfonts, pxfonts, wasysym
\usepackage{graphicx}
\usepackage{hyperref}
\usepackage{amsmath}
\usepackage{underscore}
\usepackage{caption}
\usepackage{algorithmic}
\usepackage{enumitem}
\usepackage{amssymb}
\usepackage{amsfonts}
\def\infinity{\rotatebox{90}{8}}
\begin{document}

{\centering 
\Huge And - Tree Based Search
\bigskip

\huge Search Process
\par}

\bigskip
\bigskip

Having defined the search model we are now ready to define the Search Process and Control.

\bigskip

The $And_{tree}$ will begin with a single node, $s_0 = (pr, ?)$, and its expansion will be defined by the recursive relation $Erw_{and}$. Each iteration will use $Div$ to expand the tree and create new nodes. After each $Div$, $F_{bound}$ prunes leaves that are irrelevant for our search via a branch and bound operation, using a $beta$ value. After this, $F_{leaf}$  evaluates all the leaves, and calculates a number that will correspond to the state. The search control will prioritize applying $Div$ to the lowest value leaves first. The search control will choose the left most leaf in the case that $F_{leaf}$ provides a tie between multiple leaves.
\bigskip
\bigskip

$F_{leaf}$ uses an additional helper function, $F_{penalty}$, which evaluates a penalty score of an assignment, based on the soft and hard constraints. For partial assignments, $F_{penalty}$ $^*$ is used, which uses $Eval^*$ and $Constr^*$ instead. $F_{penalty}$ is used by both $F_{leaf}$ and $F_{bound}$.

\bigskip
$F_{penalty}$ : $\{pr_1, ...$, $pr_n \} $  $ \to \mathbb{R} $    where 1 $\leq i\leq n$



 \[
          \text{F\textsubscript{penalty}} = \left\{\begin{array}{ll}
            \infinity $: 	 if $Constr(pr_i)=$ false$ \\
            Eval(pr_i) $: 	 else$\\
            
            \end{array}\right\}
      \]



 
\bigskip
Using this we can define $F_{leaf}$ : 
\bigskip

$F_{leaf}$ : $\{pr_1, ...$, $pr_n \} $  $ \to \mathbb{R} $  

$F_{leaf} = (F_{penalty}(\{pr_1, ...$, $pr_n \}))$

\bigskip

$F_{leaf}$ applies $F_{penalty}$ in order to calculate a numeric value for the search control.

\bigskip
\bigskip


$F_{bound}$ is used by the search control to keep the tree size within reason. It uses $\beta$ pruning to remove leaves that fail to beat the best found solution so far.
We can define $F_{bound}$ using $F_{bound}$. Once again we can use $F_{bound}$ $^*$ to evaluate partial assignments using $F_{penalty}$ $^*$.

\bigskip
$F_{bound}$ : $\{pr_1, ...$, $pr_n \} $  $ \to pr_i $    where 1 $\leq i\leq n$

 \[
          \text{F\textsubscript{bound}} = \left\{\begin{array}{ll}
            \beta = \infinity $: if $pr_i \in s_0 \\
           \beta = F_{penalty} $: else$\\
            
            \end{array}\right\}
      \]
      
\bigskip

$\beta _{best}$ is the smallest $\beta$ value that $F_{bound}$ or  $F_{bound}$ $^*$ has evaluated to.


if $\beta _{pr_i}$ $\leq \beta _{best}$ then $\beta_ {best} = \beta _{pr_i } $


if $\beta _{pr_i}$  $> \beta _{best}$ then  $pr_i = null$, pruning the leaf from the tree.


\bigskip
\bigskip

As there is is only one $Div$ relation, $F_{trans}$ is not used. 

There is no backtracking in this search control.

\bigskip
\bigskip
\bigskip
A state is marked as $solved$ when:

\bigskip 

$\forall X \in Prob, X_i \cup \{ \$ \} = \o $ where $X_i$ is some $X$ in $Prob$.

It is the state where each course and lab has a slot assigned to it. The state will take on the form $(pr, solved)$.
\bigskip
\bigskip
\bigskip

The search control operation operates in the following order:

\bigskip
1. Apply $F_{leaf}$ to the tree.

2. Apply $Div$ to the tree, prioritizing the branch with the lowest $F_{leaf}$ value. In case of a tie, the left most branch is used. $Div$ is applied to all unsolved branches $(pr, ?)$. It is at this point that all lower leaves are checked for $(pr, solved)$.

3. Apply $F_{bound}$ to the tree, pruning the leaves that are out of bounds if needed.

\bigskip
\bigskip
\bigskip
\bigskip
 
Search Instance:

As before the initial search state is $s_0$:

\bigskip

$s_0 = pr = <X_1, ..., X_n> $ such that $\forall X_i \in pr, X_i = \$ $ 

\bigskip

The goal state is $G_{and}$ is reached when all branches are marked with $(pr, solved)$.

\huge




 
\end{document}
