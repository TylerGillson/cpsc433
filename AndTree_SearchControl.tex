\documentclass[10pt]{article}
\usepackage[utf8]{inputenc}
\usepackage{amsfonts}% or latexsym, amssymb, mathabx, txfonts, pxfonts, wasysym
\usepackage{graphicx}
\usepackage{hyperref}
\usepackage{amsmath}
\usepackage{underscore}
\usepackage{caption}
\usepackage{algorithmic}
\usepackage{enumitem}
\usepackage{amssymb}
\usepackage{amsfonts}
\def\infinity{\rotatebox{90}{8}}
\begin{document}

{\centering 
\Huge And - Tree Based Search
\bigskip

\huge Search Process
\par}

\bigskip
\bigskip

Having defined the search model, we are now ready to define the Search Process and Search Instance.

\bigskip

The $And_{tree}$ will begin with a single node, $s_0 = (pr, ?)$, and its expansion will be defined by the recursive relation $Erw_{and}$. Each iteration will use $Div$ to expand the tree and create new nodes. After each $Div$, $F_{bound}$ evaluates all solved leaves via a branch and bound operation, using a $beta$ value. The search control uses this $\beta$ value to prune once a solution has been found. After this, $F_{leaf}$  evaluates all the leaves, and calculates a number, assessing each leaf. The search control will prioritize applying $Div$ to the lowest value leaves first. The search control will choose the left most leaf in the case that $F_{leaf}$ provides a tie between multiple leaves.
\bigskip
\bigskip

$F_{leaf}$ evaluates a penalty score of an assignment, based on the soft and hard constraints. For partial assignments, $F_{leaf}$ $^*$ is used, which uses $Eval^*$ and $Constr^*$ instead. 

\bigskip
$F_{leaf}$ : $\{pr_1$, ..., $pr_i$, ..., $pr_n \} $  $ \to \mathbb{R} $    where 1 $\leq i\leq n$



 \[
          \text{F\textsubscript{leaf}} = \left\{\begin{array}{ll}
            \infinity $: 	 if $Constr(pr_i)=$ false$ \\
            Eval(pr_i) $: 	 else$\\
            
            \end{array}\right\}
      \]

\bigskip
\bigskip


$F_{bound}$ is used by the search control to keep the tree size within reason. It sets the $\beta$ values of all complete solutions, when the state is $(pr, solved)$. 
We can define $F_{bound}$ as follows: Once again we can use $F_{bound}$ $^*$ to evaluate partial assignments.

\bigskip
$F_{bound}$ : $\{pr_1$, ..., $pr_i$, ..., $pr_n \} $  $ \to pr_i $    where 1 $\leq i\leq n$

 \[
          \text{F\textsubscript{bound}} = \left\{\begin{array}{ll}
          \beta = \beta$: if $(pr_i, ?)\\

           \beta = Eval(pr_i, ... $, $pr_n $: else$\\
           
            
            \end{array}\right\}
      \]
      
\bigskip

$\beta _{best}$ is the smallest $\beta$ value that $F_{bound}$.

if $\beta _{pr_i}$ $\leq \beta _{best}$ then $\beta_ {best} = \beta _{pr_i } $





\bigskip
\bigskip

As there is is only one $Div$ relation, $F_{trans}$ is not used. 

There is no backtracking in this search control.

\bigskip
\bigskip
\bigskip

The search control operation operates in the following order:

\bigskip


1. Apply $Div$ to the tree, prioritizing the branch with the lowest $F_{leaf}$ value. In case of a tie, the left most branch is used. 

2. Apply $F_{bound}$.  

3. Apply $F_{leaf}$. If $F_{leaf} \geq \beta $, then prune the leaf.

4. We check states to see if they are solved: We mark a state as $(pr, solved)$ if $\forall X \in pr, X \neq \$ $ and $F_{leaf}(pr_1, ... $, $pr_n) \neq \infinity$


5. We choose the smallest $F_{leaf}$ value of the leaves that are marked $(pr, ?)$  and apply $Div$ again, starting the process over again.

\bigskip
\bigskip
\bigskip
\bigskip
 
Search Instance:

As before the initial search state is $s_0$:

\bigskip

$s_0 = pr = <X_1, ..., X_n> $ such that $\forall X_i \in pr, X_i = \$ $ 
or $s_0$ is some partial assignment given as an input to the search.

We also set $\beta = \infinity$.
\bigskip

The goal state is $G_{and}$ is reached when all leaf nodes are marked with $(pr, yes)$.

\bigskip

The optimal solution is the leaf node with the lowest $F_{leaf}$ value.
\huge




 
\end{document}
